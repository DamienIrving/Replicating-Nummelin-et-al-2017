\section{Surface heat flux}

My surface heat flux curve (Figure \ref{fig:hfds}) has a similar shape to Figure 1b of \citet{Nummelin_2017} (any differences are probably just explained by the fact that I haven't divided by area) but the magnitude seems too small relative to the ocean heat transport convergence curve.

When the \textbf{calc_surface_forcing_maps.py} script is passed a monthly timescale hfds file, it does the following:
\begin{enumerate}
\item Converts the data to an annual timescale
\item Multiplies each grid point value by its corresponding area (from the areacello file; this converts the units from $Wm^{-2}$ to $W$)
\item Regrids to a regular lat/lon grid (if necessary)
\item Calculates the (unweighted) zonal sum
\end{enumerate}

\begin{verbatim}
$ python ~/ocean_analysis/data_processing/calc_surface_forcing_maps.py
hfds_Omon_NorESM1-M_rcp85_r1i1p1_200601-210012.nc 
surface_downward_heat_flux_in_sea_water 
hfds-by-areacello-zs_Oyr_NorESM1-M_rcp85_r1i1p1_all.nc 
--area_file areacello_fx_NorESM1-M_rcp85_r0i0p0.nc
\end{verbatim}

I went for the zonal sum as opposed to zonal mean because I figured the hfbasin data represents a zonal sum (i.e. the accumulated heat transport across a given latitude circle).

The \textbf{plot_heat_trends.py} script then calculates the linear trend at each latitude and plots the result. I'm at a loss to explain why the amplitude of the plotted trends are slightly too small relative to the ocean heat transport convergence trends.



