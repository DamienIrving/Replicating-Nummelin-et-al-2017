As shown in Figure \ref{fig:noresm_full}, I'm seeing a bit of a discrepancy between the ocean heat content tendency and that inferred by the sum of surface heat flux and heat transport convergence. This could be due to (a) the regridding, (b) the polynomial fitting approach used to calculate the ocean heat content tendency (i.e. the assumptions underlying that approach), or (c) some other reason. To get a feel for how influential the regridding is, I've generated the same plot using CSIRO-Mk3-6-0 and GISS-E2-R data, since the ocean data from those models is on a regular lat/lon grid (bottom panel of Figures \ref{fig:csiro} and \ref{fig:giss}), and have also analysed NorESM1-M on its native grid (Figure \ref{fig:noresm}). I've also included a simple analysis of the mean heat budget in the top panel of these plots, just to see how close to balanced the heat budget is in each model.