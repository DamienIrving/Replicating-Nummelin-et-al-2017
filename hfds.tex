\section{Appendix: Surface heat flux methods}

When the \textbf{calc_surface_forcing_maps.py} script is passed a monthly timescale hfds file, it does the following:
\begin{enumerate}
\item Converts the data to an annual timescale
\item Regrids to a regular lat/lon grid (if necessary)
\item Multiplies each grid point value by its corresponding area (from the areacello file; this converts the units from $Wm^{-2}$ to $W$)
\item Calculates the (unweighted) zonal sum
\end{enumerate}

\begin{verbatim}
$ python ~/ocean_analysis/data_processing/calc_surface_forcing_maps.py
hfds_Omon_NorESM1-M_rcp85_r1i1p1_200601-210012.nc 
surface_downward_heat_flux_in_sea_water 
hfds-by-areacello-zs_Oyr_NorESM1-M_rcp85_r1i1p1_all.nc 
--area
\end{verbatim}

Ordinarily I would have multiplied by area on the native grid (using the areacello data) before regridding, however for some reason the iris regridding function gives crazy values if I do this for the NorESM1-M model (not sure why - it doesn't do this in other situations). 

The \textbf{plot_heat_trends.py} script then calculates the linear trend at each latitude and plots the result.
