\section{Appendix: Ocean heat transport convergence methods}

When the \textbf{calc_ocean_heat_transport_convergence.py} script is passed a hfbasin file, it simply calculates the difference between adjacent values using \textbf{numpy.diff} and then multiplies the result by -1. e.g:

\begin{verbatim}
$ python ~/ocean_analysis/data_processing/calc_ocean_heat_transport_convergence.py
hfbasin_Omon_NorESM1-M_rcp85_r1i1p1_200601-210012.nc
northward_ocean_heat_transport
hfbasin-convergence_Omon_NorESM1-M_rcp85_r1i1p1_all.nc
\end{verbatim}

The reason for the multiplication by -1 is that \textbf{numpy.diff} calculates $diff_{n} = value_{n+1} - value_{n}$, whereas the calculation used in \citet{Nummelin_2017} is $diff_{n} = value_{n} - value_{n+1}$.    

If \textbf{calc_ocean_heat_transport_convergence.py} is passed a hfy file on a lat/lon grid, it simply calculates the zonal sum at each latitude (i.e. to create a hfbasin-equivalent variable) prior to proceeding as above. 

The \textbf{plot_heat_trends.py} script then converts the monthly timeseries to an annual timeseries and calculates the linear trend at each latitude and plots the result. 